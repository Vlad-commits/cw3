\documentclass[%
bachelor,    % тип документа
%natbib,      % использовать пакет natbib для "сжатия" цитирований
subf,        % использовать пакет subcaption для вложенной нумерации рисунков
href,        % использовать пакет hyperref для создания гиперссылок
colorlinks,  % цветные гиперссылки
%fixint,     % включить прямые знаки интегралов
]{disser}

\usepackage[
  a4paper, mag=1000,
  left=2.5cm, right=1cm, top=2cm, bottom=2cm, headsep=0.7cm, footskip=1cm
]{geometry}


\usepackage[intlimits]{amsmath}
\usepackage{amssymb,amsfonts}

\usepackage[T2A]{fontenc}
\usepackage[utf8]{inputenc}
\usepackage{pgf}
\usepackage[english,russian]{babel}
\ifpdf\usepackage{epstopdf}\fi
\usepackage[autostyle]{csquotes}

% Шрифт Times в тексте как основной
%\usepackage{tempora}
% альтернативный пакет из дистрибутива TeX Live
%\usepackage{cyrtimes}

% Шрифт Times в формулах как основной
%\usepackage[varg,cmbraces,cmintegrals]{newtxmath}
% альтернативный пакет
%\usepackage[subscriptcorrection,nofontinfo]{mtpro2}

\usepackage[%
  style=gost-numeric,
  backend=biber,
  language=auto,
  hyperref=auto,
  autolang=other,
  sorting=none
]{biblatex}

\addbibresource{report.bib}

% Плавающие рисунки "в оборку".
\usepackage{wrapfig}

% Номера страниц снизу и по центру
%\pagestyle{footcenter}
%\chapterpagestyle{footcenter}

% Точка с запятой в качестве разделителя между номерами цитирований
%\setcitestyle{semicolon}

% Точка с запятой в качестве разделителя между номерами цитирований
%\setcitestyle{semicolon}

% plots fixes
%\usepackage{graphicx}
%\usepackage{epstopdf}
%\usepackage{pgfplots}
%\usepackage{tikz}
%\usepgfplotslibrary{external} 
%\usetikzlibrary{external}
%\tikzexternalize

% Использовать полужирное начертание для векторов
\let\vec=\mathbf

% Включать подсекции в оглавление
\setcounter{tocdepth}{2}

\graphicspath{{fig/}}

\DeclareMathOperator*{\argmin}{argmin}
\DeclareMathOperator*{\argmax}{argmax}

\usepackage{afterpage}
\usepackage{amsthm}

\begin{document}

% Переопределение стандартных заголовков
%\def\contentsname{Содержание}
%\def\conclusionname{Выводы}
%\def\bibname{Литература}

%
% Титульный лист на русском языке
%

\institution{Санкт-Петербургский государственный университет}


\title{Курсовая работа}

% TODO
\topic{Сравнительный анализ систем загрузки больших данных}

% Автор
\author{Мамаев Владислав Викторович}
% Группа
\group{323}
% Номер специальности
\coursenum{010400 (01.03.02)}
% Название специальности
\course{Прикладная математика и информатика}

% Научный руководитель

\sa      {Благов Михаил Валерьевич}
% TODO
\sastatus{к.~ф.-м.~н., }
% Город и год
\city{Санкт-Петербург}
\date{\number\year}

\maketitle

%%
%% Titlepage in English
%%
%
%\institution{Name of Organization}
%
%% Approved by
%\apname{Professor S.\,S.~Sidorov}
%
%\title{Bachelor's Thesis}
%
%% Topic
%\topic{Dummy Title}
%
%% Author
%\author{Author's Name} % Full Name
%\course{Physics} % Specialization
%
%\group{} % Study Group
%
%% Scientific Advisor
%\sa       {I.\,I.~Ivanov}
%\sastatus {Professor}
%
%% Reviewer
%\rev      {P.\,P.~Petrov}
%\revstatus{Associate Professor}
%
%% Consultant
%\con{}
%\conspec{}
%\constatus{}
%
%% City & Year
%\city{Saint Petersburg}
%\date{\number\year}
%
%\maketitle[en]

% Содержание
\tableofcontents


\section{Введение}
%Что такое OLTP и OLAP? Для чего нужны?
  
Базы данных используются повсеместно, при этом системы, для которых они предназначены, можно разделить на 2 класса:
\begin{itemize}
\item Online Transaction Processing (OLTP) -- системы обработки транзакций в реальном времени. Такие системы используются для операционной деятельности предприятий.
\item Online Analytical Processing (OLAP) -- системы аналитической обработки в реальном времени. К такими системам относятся системы поддержки принятия решений, инструменты business intelligence, системы анализа данных.
\end{itemize}

% Почему их следует разделять? 
OLAP системы содержат информацию из OTLP систем, при обычно OLAP системы обычно функционирует независимо от OLTP систем по следующим причинам: \cite{oltp_olap}
\begin{itemize}
	\item Реляционная схема данных, обычно используемая в OLTP системах, не эффективна для OLAP нагрузки.
	\item OLAP нагрузка на OLTP систему может вызвать проблемы с производительностью транзакций.
	\item С увеличением размера хранимых данных, возникают ограничения на используемые технологии и существенно увеличивается стоимость хранения данных в системах не предназначенных только для OLAP.
	\item Необходимость доступа к данным из разных OLAP систем и возможности ограничения доступа к данным
\end{itemize}

% Какие проблемы при интеграции?
Традиционно, данные попадают в OLAP систему из OLTP системы при помощи Extraction-Transformation-Loading (ETL) программ,
запускаемых с некоторой периодичностью. ETL процессы могут занимать достаточно много времени, и это создает задержку появления данных в OLAP системе. 

Для того чтобы уменьшить задержку можно вместо переодичных ETL процессов использовать потоковую обработку.
Подобная архитектура интеграции данных описана в \cite{streaming_integration} и представленна на рисунке TODO.
Каждое изменение данных в OLTP системе записывается в очередь сообщений, а некоторая программа непрерывно читает сообщения и обновляет данные в аналитическом хранилище.
Использование непрерывных обновлений уменьшает задержку, но требует специальных инструментов для обеспечения целостности данных --- традицонно используемые хранилища для большого объема данных, такие как Hadoop, работают по принципу write-once и не поддерживают обновлений. 

В данной работе будет проведено сравнение таких инструментов: Apache Hudi, Delta Lake, Apache Iceberg.
Сравнение будет произведино по следующим пунктам:
\begin{itemize}
	\item Время задержки доставки данных.
	\item Пропускная способность.
	\item Удобство доступа к данным --- наличие интеграций с другими инструментами для работы с большими данными и совместимость.
	\item Безопасность.
\end{itemize}
\section{Основная часть}

\cite{test_data}

\subsection{Основная часть}




% Список литературы
\printbibliography[heading=bibintoc]

% Приложения
\appendix


\end{document}